\documentclass[12pt]{article} %{{{
\usepackage[margin=1in]{geometry}

% Figures
\usepackage{graphicx}
\graphicspath{{../../plots/}}

% Math
\usepackage{amsmath}
\usepackage{amssymb}
\DeclareMathOperator*{\argmin}{\arg\!\min}
\DeclareMathOperator*{\argmax}{\arg\!\max}

% abbreviations
\def\eg{e.g.,~}
\def\ie{i.e.,~}
\def\cf{cf.\ }
\def\viz{viz.\ }
\def\vs{vs.\ }

% Refs
\usepackage[style=nature, backend=bibtex]{biblatex}
\addbibresource{main.bib}

\usepackage{url}

\newcommand{\secref}[1]{Section~\ref{sec:#1}}
\newcommand{\figref}[1]{Fig.~\ref{fig:#1}}
\newcommand{\tabref}[1]{Table~\ref{tab:#1}}
%\newcommand{\eqnref}[1]{\eqref{eq:#1}}
%\newcommand{\thmref}[1]{Theorem~\ref{#1}}
%\newcommand{\prgref}[1]{Program~\ref{#1}}
%\newcommand{\algref}[1]{Algorithm~\ref{#1}}
%\newcommand{\clmref}[1]{Claim~\ref{#1}}
%\newcommand{\lemref}[1]{Lemma~\ref{#1}}
%\newcommand{\ptyref}[1]{Property~\ref{#1}}

% for quick author comments 
\usepackage[usenames,dvipsnames,svgnames,table]{xcolor}
\definecolor{light-gray}{gray}{0.8}
\def\del#1{ {\color{light-gray}{#1}} }
\def\yy#1{\footnote{\color{red}\textbf{yy: #1}} }

% subfigures
\usepackage{caption}
\usepackage{subcaption}

%}}}

\begin{document} %{{{

\title{Information theory for discrete modelling} %{{{
\date{\today}
\maketitle %}}}

\section{Introduction}\label{sec:introduction} %{{{

%}}}

\section{Boolean Minimization}\label{sec:boolmin} %{{{ 

Boolean functions consist of a set of inputs that take boolean values and 
collectively determine the transition of the overall function. The function 
itself also takes a boolean value as its state and can transition between these
state values. The rule by which the function transitions is often described in a
look up table (LUT). An example of such a LUT can be seen in 
\figref{parity_lut}. [I supose I should give a proper treatment to what exactly
a boolean automaton is here. That can be filled in later once I have the 
skeleton laid out.]

\begin{figure}[h]
    \centering
    \begin{subfigure}[A]{0.48\textwidth}
        \centering
        \includegraphics[height=6cm]{example_luts_schemata/parity_lut.pdf}
        \label{fig:parity_lut}
        \caption{\label{fig:parity_lut}}
    \end{subfigure}
    \begin{subfigure}[B]{0.48\textwidth}
        \centering
        \includegraphics[height=6cm]{example_luts_schemata/parity_ts.pdf}
        \label{fig:parity_ts}
        \caption{\label{fig:parity_ts}}
    \end{subfigure}
    \caption{Visual descriptions of three bit parity: 
    (\subref{fig:parity_lut}) Unmodified look up table (LUT);
    (\subref{fig:parity_ts}) Minimized schemata.}
    \label{fig:parity}
\end{figure}

Boolean minimization is a process by which the number of rows in the LUT of a
boolean function is minimized. Typically, this is achieved by replacing 
non-informative input states with ``wildcards" (shown as \# in the schemata in
\figref{parity_ts}) that represent an indifference to the actual state of that 
input in determining the transition. This process can be achieved efficiently
through the Quine-McCluskey method of prime implicants \cite{quine_way_1955}.

Boolean functions can be further reduced by leveraging the observation that in
many cases it does not matter which input adopts a particular state. In the case
of the parity function shown in \figref{parity} it does not matter which input 
is in the ON (1) state so long as only one input is in the ON state. The inputs 
can be symmetrical with respect to one another's states 
\cite{marques-pita_canalization_2013}. This can be seen very 
obviously in the \textit{two-symbol schemata} for the three bit parity function
shown in \figref{parity_ts}. The name \textit{two-symbol} refers to the use
of a second, ``position-free," symbol $^\circ$.

These redescriptions highlight the role of boolean functions as models of 
information processing elements. \textit{Boolean networks} composed of many
boolean functions, each serving as inputs to other functions in the network are
canonical models of complex systems, particularly those involved in information
processing in biological systems \cite{kauffman_emergent_1984,willadsen_robustness_2007,marques-pita_canalization_2013} 
[get more of these into the .bib probably]. Despite their use in modelling
information processing, formal measures from information theory lag behind. This
stems largely from the fact that measures based on Shannon entropy are unsuited
to dealing with polyadic relationships 
\cite{james_information_2016,james_multivariate_2017} such as the ones almost
universally present in boolean functions. 
%}}}

\section{Partial Information Decomposition}\label{sec:partial}

Partial Information Decomposition (PID) is a framework originally proposed by
Williams and Beer \cite{williams_nonnegative_2010}. In this framework the
total information between a set of source varibles ($S = \{X_1, X_2\}$) and a 
target variable ($T$) are decomposed into different non-overlapping pieces. 
\textit{Redundancy} is information that could be provided by any variable in 
the set of sources. 

\section{Methods}\label{sec:methods} %{{{

%}}}

\printbibliography
    
\end{document} %}}}
